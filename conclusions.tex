\section{Conclusions}
\label{sec:conclusions}
Processes that involve vertices of three and four electroweak gauge bosons are among the most promising processes
where new physics might be found. Deviations from the Standard Model couplings or vertices that do not exist in the 
Standard Model might be found. The deviations from the Standard Model predictions can for instance conveniently be 
described by higher dimensional operators in terms of an effective field theory. Precise measurements of these  
processes allow to constrain and set limits to these higher dimensional operators.

In this paper we investigated the Standard Model predictions to a subset of such processes where we require
two photons plus an additional electroweak vector boson in the final state.  As additional vector boson we allowed
for a third photon as well as for a $W$ or $Z$ boson where we considered their leptonic decay modes.
This particularly implies that all off-shell and non-resonant contributions are taken into account.

We calculated the next-to-leading order QCD and electroweak corrections to these three processes for a set
of realistic fiducial cuts. Particular
emphasis has been put on the up to now unknown electroweak corrections as well as the combination of QCD and
electroweak corrections with the aim of producing the most precise prediction possible within the Standard Model.
As expected, we found the QCD corrections to be large and dominant compared to the electroweak corrections when 
calculating the corrections to the total cross sections. QCD corrections are particular large for these processes due to new
channels opening up at NLO but can effectively be reduced by applying a jet veto. Electroweak corrections
lead to moderate corrections to the total cross section of up to $4.4\%$ for the $\gamma\gamma \ell^{+}\ell^{-}$ 
process. However, in the high energy tail of differential distributions the electroweak corrections become increasingly important
and can lead to corrections of up to $30 \%$. Electroweak corrections become important in the same regions where one
also expects increasing effects of higher dimensional operators. The precise determination of limits on higher dimensional 
operators therefore requires the inclusion of higher order correction of both QCD as well as of the electroweak interaction.

We also compared the two possibilities of combining QCD with electroweak corrections, namely either in an additive or
in a multiplicative way. One expects that the difference between the two schemes is small as they can be seen as an estimation
of neglected higher order contributions. We found that although this is true on the level of total cross sections there can be 
regions in phase space where the leading order contribution vanishes and therefore one or both $K$-factors become infinitely large. 
For those kinematical edges we observe a breakdown of the multiplicative scheme whereas the additive scheme provides
a reliable estimation of the higher order uncertainties throughout the whole phase space.




