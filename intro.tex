\section{Introduction}
\label{sec:intro}

The precise understanding of the electroweak symmetry breaking mechanism is an important cornerstone in verifying the
Standard Model. Different realizations of the electroweak sector or new physics coupling to electroweak gauge bosons will
yield to deviations compared to the Standard Model prediction. New physics effects can conveniently be described in terms
of an effective theory where new heavy degrees of freedom are integrated out and deviations from the Standard Model are 
parametrised by higher dimensional operators \cite{Weinberg:1978kz,Weinberg:1979pi}. Higher dimensional operators can lead
 to deviations in triple and quartic gauge couplings. Moreover, new vertices (e.g. $Z\gamma\gamma$) that do not exist in the 
Standard Model can appear. The class of processes that involve a pair of photons in association with another vector boson
allows to measure deviations in triple and quartic gauge couplings and is therefore a particularly interesting class of processes.
 Consequently both ATLAS and CMS have measured these type of processes und derived constraints on anomalous gauge
 couplings \cite{Aad:2016sau,Sirunyan:2017lvq,Aad:2015uqa,Aad:2015bua}.\\
 In this paper we calculate the next-to-leading order QCD and electroweak corrections to the processes 
 $\gamma\gamma\gamma$, $\gamma\gamma e^{+} e^{-}$ and $\gamma\gamma e^{-} \bar{\nu_e}$. For the latter two, all
 possible off-shell contributions are taken into account. The QCD corrections to these processes are already known 
 \cite{Bozzi:2011en,Campbell:2012ft,Bozzi:2011wwa}. For a more complete picture of the higher order effects we also
recalculate the QCD corrections for a center of mass energy of $13$ TeV, and supplement them with the 
 next-to-leading order electroweak corrections. Although the electroweak corrections are much smaller at the level
 of the total cross section compared to the NLO QCD corrections they are particularly important when deriving limits on 
 anomalous couplings. The effects of higher dimensional operators increase in the high energy tails of differential distributions
 and lead to a change of the shapes. And it is in the same region where the Sudakov logarithms from the electroweak
 corrections will play an essential role as well.\\
 The paper is organized as follows. In section \ref{sec:setup} we describe the calculational setup that has been used to obtain our
 numerical result which we are going to discuss in section \ref{sec:results} before we conclude in section \ref{sec:conclusions}.
 