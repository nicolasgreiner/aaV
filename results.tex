\section{Results}
\label{sec:results}

In this section we present numerical results for the NLO QCD and NLO EW 
corrections to all three production processes of a diphoton pair in 
association with a third vector boson, a third photon, a $W$ or a $Z$ 
boson, at the LHC at a centre-of-mass energy of 13\,TeV. 
In case of an accompanying $W$ or a $Z$ boson, we consider the full 
off-shell leptonic final state, i.e.\ lepton-neutrino or lepton-pair 
production.
All results are obtained in the Standard Model using the complex-mass 
scheme \cite{Denner:2005fg} with the following input parameters
\begin{center}
  \begin{tabular}{rclrcl}
    $\alphazero$ &\shortequal& $1/137.03599976$  \qquad &&& \\
    $\Gmu$ &\shortequal& $1.1663787\times 10^{-5}\; \text{GeV}^2$ &&& \\
    $m_W$ &\shortequal& $80.385\; \text{GeV}$       & $\Gamma_W$ &\shortequal& $2.085\; \text{GeV}$ \\
    $m_Z$ &\shortequal& $91.1876\; \text{GeV}$      & $\Gamma_Z$ &\shortequal& $2.4952\; \text{GeV}$ \\
    $m_h$ &\shortequal& $125.0\; \text{GeV}$        & $\Gamma_h$ &\shortequal& $0$\\
    $m_t$ &\shortequal& $173.2\; \text{GeV}$        & $\Gamma_t$ &\shortequal& $0$\;.
  \end{tabular}
\end{center}
While we calculate triple photon production in the \alphazero-scheme, 
we use a mixed scheme for \aaw\ and \aaz production: at LO two 
powers of $\alpha$ are taken in the \alphazero-scheme, while 
one power of $\alpha$ is taken in the \Gmu-scheme. 
The additional power of $\alpha$ in the NLO EW correction 
is evaluated in \alphazero-scheme again.
The virtual amplitudes are renormalised correspondingly.
In all cases both the width of the top quark and the Higgs boson 
can safely be neglected as there are no diagrams containing either 
as $s$-channel propagators which can potentially go on-shell. 
All other lepton and parton masses and widths are set to zero, 
i.e.\ we are working in the five-flavour scheme.
We use the \textsc{CT14nlo} PDF set with $\alphas(m_Z)=0.118$ 
\cite{Dulat:2015mca}, interfaced through LHAPDF6 \cite{Buckley:2014ana}. 
The use of a QCD-only PDF is justified by the fact that, 
at LO, the photon induced corrections are either non-existant 
(\aaa, \aaw) or negligible (\aaz).
This finding will be detailed in Section \ref{sec:results:aaz}.

We define our central scales through
\begin{equation}
  \label{eq:murfdef}
  %\begin{split}
    \muR^0 = \muF^0 = %\left\{
    %\begin{array}{ll}
     % m_{\aaa} \qquad & \text{in \aaa\ production} \\[2mm]
      \tfrac{1}{2}\,H_T' \;. %& \text{in \aaw\ and \aaz\ production.}
    %\end{array}\right.
  %\end{split}
\end{equation}
In the case of the triple photon process $ H_\mathrm{T}' $ is just given by the scalar sum of all final state 
transverse momenta, for the two processes with the massive vector bosons it is defined as 
\begin{equation}
  \label{ew:defHT}
  \begin{split}
    H_\mathrm{T}' = E_\mathrm{T}^V + \sum_{\gamma,q,g} p_{\mathrm{T},i}
  \end{split}
\end{equation}
with $\left.E_\mathrm{T}^W\right.^2=(p_\ell+p_\nu)^2$ and 
$\left.E_\mathrm{T}^Z\right.^2=(p_{\ell^+}+p_{\ell^-})^2$ 
in full analogy to the case of the case of vector boson production 
in association with jets \cite{Berger:2009ep}.
As the Born process in each case has no $\muR$ dependence, we 
do not expect the choice of scale to have a significant influence 
on the size of the relative QCD and EW corrections.
We calculate the leading order cross section $\rd\sigma_\text{LO}(\muF)$, 
which only depends on the factorisation scale $\muF$, the 
NLO QCD differential correction factor $\deltaQCD(\muR,\muF)$, 
introducing the additional $\muR$-dependence, and the NLO EW 
differential correction factor $\deltaEW(\muF^0)$. 
To estimate the impact of yet-to-be-calculated higher-order 
corrections we vary the free scales $\muR$ and $\muF$ 
by the conventional factor of two around their central values 
$\muR^0$ and $\muF^0$, respectively.
We do not vary the factorisation scale for the determination 
of $\deltaEW$ as the inherent, albeit normally phenomenologically 
irrelevant, stabilisation of the $\muF$-dependence at NLO EW 
is not reflected in our chosen PDF.
Hence, our NLO EW result exhibits the exact same $\muF$-dependence 
as the LO result.
We define 
\begin{equation}
  \label{eq:defnlo}
  \begin{split}
    \rd\sigma_\text{NLO QCD}
    \,=\;&\rd\sigma_\text{LO}\left(1+\deltaQCD\right)\\
    \rd\sigma_\text{NLO EW}
    \,=\;&\rd\sigma_\text{LO}\left(1+\deltaEW\right)\\
    \rd\sigma_\text{NLO \QCDpEW}
    \,=\;&\rd\sigma_\text{LO}\left(1+\deltaQCD+\deltaEW\right)\\
    \rd\sigma_\text{NLO \QCDtEW}
    \,=\;&\rd\sigma_\text{LO}\left(1+\deltaQCD\right)\left(1+\deltaEW\right)\;.
  \end{split}
\end{equation}
Therein, the difference between NLO \QCDpEW\ and NLO \QCDtEW, which 
is of relative $\order(\alphas\alpha)$, can serve as an indicator 
of the potential size of unknown corrections at that order.

In Table \ref{tab:xsec} we quote the inclusive cross sections for all 
three processes. 
The set of fiducial cuts for each process is detailed in its 
respective subsection below. 
We note that the NLO QCD corrections for both triple photon 
production and \aaw\ production are strongly jet veto dependent, 
a result that was previously discussed in great detail in 
\cite{Bozzi:2011en,Bozzi:2011wwa}, and will be revisited in the 
following. 
A much milder jet veto dependence is found for \aaz\ production. 
The electroweak corrections to inclusive cross sections are 
generally much smaller, ranging from $0.6\%$ (\aaa) to $-1.8\%$ (\aaw) 
and $-4.4\%$ (\aaz).

\begin{table}[t!]
  \centering
\begin{tabular}{l||c|c|c}
  & $\;\;pp \to \gamma \gamma\gamma\;\;$
  & $\;\;pp \to \gamma \gamma e^-\bar\nu_e\;\;$
  & $\;\;pp \to \gamma \gamma e^+e^-\;\;$ \\
  \hline\hline
  $\sigma_\text{LO}\;\;[\text{fb}]\vP$ & $5.56_{-0.36}^{+0.30}$ & $0.92_{-0.07}^{+0.06}$ & $4.21_{-0.41}^{+0.36}$ \\
  \hline\hline
  $\delta_\text{QCD}\;\;[\%]\;\;\pTveto=\infty\vP$ & $139_{-27}^{+24}$ & $111_{-24}^{+21}$ & $27_{-18}^{+13}$ \\
  \hline
  $\delta_\text{QCD}\;\;[\%]\;\;\pTveto=30\,\text{GeV}\vP$ & $\hspace*{\unitcharwidth}35_{-13}^{+\hspace*{\unitsuperscriptwidth}7}$ & $\hspace*{\unitcharwidth}41_{-14}^{+\hspace*{\unitsuperscriptwidth}8}$ & $19_{-17}^{+11}$ \\
  \hline
  $\delta_\text{EW}\;\;[\%]\vP$ & $0.6$ & $-1.8$ & $-4.4$ \\
\end{tabular}
  \caption{
    Total cross sections at LO, NLO QCD and NLO EW for $\gamma\gamma\gamma$, 
    $\gamma\gamma e^-\bar\nu_e$ and $\gamma\gamma e^+e^-$
    production at 13\,TeV at the LHC.
    \label{tab:xsec}
  } 
\end{table}


\subsection[\texorpdfstring{$\gamma\gamma\gamma$}{aaa} production]
           {$\boldsymbol{\gamma\gamma\gamma}$ production}
\label{sec:results:aaa}

The triple photon production process is defined 
by the presence of three identified photons in the 
central detector. 
To this end we use the smooth cone isolation 
criterion \cite{Frixione:1998jh}, limiting the amount of 
hadronic activity in a cone $R_\gamma$ to
\begin{equation}
  \begin{split}
    E_{{\rm had, max}} (r_{\gamma})
    = \epsilon\, \pT^{\gamma} \left( \frac{1-\cos r_\gamma}
				       {1-\cos R_\gamma}\right)^{n}\;,
  \end{split}
  \label{eq:frix}
\end{equation}
where $r_{\gamma}$ denotes the angular separation between the photon and 
the parton, with 
\begin{equation}
  \label{eq:coneparams}
  \begin{split}
    R_{\gamma}=0.4\;, \quad \epsilon = 0.05\;, \quad n = 1\;, 
  \end{split}
\end{equation}
to define isolated photon candidates.
These candidates are then ordered in transverse momentum. 
We require at least three such candidates within $|\eta|<2.37$, 
the leading one of which needs $\pT>40\,\text{GeV}$, 
while the all subleading ones need only 
$\pT>30\,\text{GeV}$. 
Finally, a pairwise separation of $\Delta R(\gamma_i,\gamma_j)>0.4$ 
between all identified photons is required.
It is worth noting that at NLO EW it is possible to find more 
than three isolated photons, in which case any combination may 
fulfill the above criteria. 

\begin{figure}[t!]
  \centering
  \includegraphics[width=0.32\textwidth]{figs_aaa/pT_y1}
  \includegraphics[width=0.32\textwidth]{figs_aaa/pT_y2}
  \includegraphics[width=0.32\textwidth]{figs_aaa/pT_y3}
  \caption{
    Transverse momentum of the leading (left), subleading (centre) 
    and third leading (right) photon 
    in triple photon production at the LHC at 13\,TeV. 
    The distributions are shown at LO (blue), NLO QCD (green), 
    NLO EW (orange), NLO \QCDpEW\ (black) and NLO \QCDtEW\ (red) 
    including scale uncertainties. The top ratio plot details 
    the relative corrections to the leading order cross section 
    without applying any jet veto, while the centre ratio plot 
    applies a jet veto of $p_\mathrm{T,jet}^\mathrm{veto}=\text{30\,GeV}$. 
    The lower ratio highlights the size of the electroweak corrections.
    \label{fig:aaa:pt}
  }
\end{figure}

Figure \ref{fig:aaa:pt} displays the transverse momenta of the 
first three leading photons. 
The first observation is that, with the chosen set of cuts, 
the leading and subleading photon are of similar hardness, 
and generally much harder than the third leading photon. 
This suggest that we select primarily diphoton production events 
which is accompanied by a third, mostly bremsstrahlung, photon. 
The NLO QCD corrections exhibit a handful of interesting features. 
In the absence of a jet veto the fixed-order calculation 
exhibits huge correction factors, mainly induced by the 
opening of new channels in the real emission. 
Additionally, kinematic constraints present at LO \footnote{
  At leading order the leading photon needs to be in a different 
  hemisphere than both the subleading and third leading photon. 
  Thus, $\Delta\phi_{\gamma_1\gamma_2}$ and $\Delta\phi_{\gamma_1\gamma_3}$ 
  must be larger that $\tfrac{1}{2}\,\pi$.
} are released and 
lead to a larger phase space that can be populated. 
These findings mandate the inclusion of at least the $\aaa+\text{jet}$ 
production process at NLO QCD to arrive at a reliable description 
of inclusive \aaa\ production, and thus either a NNLO QCD calculation 
or a multijet merging ansatz \cite{Hoeche:2012yf,Kallweit:2015dum}. 
While the inclusive QCD corrections at very small transverse momenta 
are universally large for all three leading photon \pT-spectra 
($\deltaQCD\approx 2$), they remain at exorbitantly large 
($\deltaQCD\approx 1.5$) throughout the considered range 
only for the leading jet \pT. 
For both subleading photons the QCD corrections are quickly 
decreasing, leveling out at a approximately $20\%$ at large transverse 
momenta.
In the presence of a restrictive jet veto the very low transverse 
momentum region still experiences large correction of about 
$\deltaQCD\approx 1$. 
As transverse momenta are increasing, however, the QCD corrections now turn 
negative reaching now $-50-60\%$ and for all three photons. 

The electroweak corrections, due to the absence of the opening 
of large new channels at the next-to-leading order, are dominated 
by the virtual corrections. 
Consequently, the release of the LO phase space restrictions 
in the real emission corrections only plays a minor role.
For all three photons the electroweak corrections are small but 
positive at small transverse momenta and exhibit the usual 
Sudakov shape at large transverse momenta.
They reach $-10\%$ for the leading and subleading photon and 
$-20\%$ for the third photon at $\pT=500\,\text{GeV}$. 
More importantly, beyond $\pT\gtrsim 170$, 140 and 80\,GeV 
for the first, second and third leading photon, respectively, 
the electroweak corrections are not covered by the LO 
uncertainty estimate.
Due to the different sizes of the QCD and electroweak 
corrections, the additive and multiplicative combination 
of corrections lead to very similar results.

\begin{figure}[t!]
  \centering
  \includegraphics[width=0.32\textwidth]{figs_aaa/m_y1y2_comb_log}
  \includegraphics[width=0.32\textwidth]{figs_aaa/m_y1y3_comb_log}
  \includegraphics[width=0.32\textwidth]{figs_aaa/m_y2y3_comb_log}
  \caption{
    Pairwise invariant mass of the leading and subleading photon (left),
    leading and third leading photon (centre), subleading and third leading 
    photon (right) 
    in triple photon production at the LHC at 13\,TeV. 
    Details as in Fig.\ \ref{fig:aaa:pt}.
    \label{fig:aaa:myy}
  }
\end{figure}

Figure \ref{fig:aaa:myy} continues with the three combinations of 
diphoton invariant masses. 
The \pT\ and $\Delta R$ requirements 
of the event selection induce a minimum in the distributions 
at leading order. 
The region below can only be filled if a fourth particle is 
present, as is the case in both the QCD and electroweak real 
emission corrections, leading to simultaneously huge 
corrections \deltaQCD\ and \deltaEW\ as the Born cross 
section vanishes. 
Due to this behaviour, the multiplicative combination of 
corrections, NLO \QCDtEW, ceases to be well defined and 
spikes in the distribution are visible.
The distributions below, where $\rd\sigma_\text{LO}=0$, 
are ill-defined. 
In consequence, for distributions where kinematic boundaries 
exist at leading order, but are lifted at higher orders, 
the multiplicative combination does not present a viable 
option for describing the observable 
throughout phase space. 

The QCD corrections themselves again exceed $200\%$ at 
small invariant masses, already well before the above 
described kinematic boundary effect takes hold. 
As the invariant masses are increasing, the 
QCD corrections are dropping to $20-30\%$ for the inclusive 
selection. 
The structure the QCD corrections exhibit around 80-90\,GeV 
in all three diphoton-pair invariant masses are induced by 
the acceptance cuts.
In the presence of the jet veto, the QCD corrections are 
reduced and turn negative beyond $m_{\gamma\gamma}\gtrsim 300\,\text{GeV}$ 
reaching around $-40\%$ at 1\,TeV. 
The electroweak corrections, on the other hand, are 
again moderate, ranging from $+1.5\%$ between 70 and 200\,GeV 
for $m_{\gamma_1\gamma_2}$ and $m_{\gamma_1\gamma_3}$ and 
0 and 100\,GeV for $m_{\gamma_2\gamma_3}$. 
$m_{\gamma_1\gamma_2}$ exhibits a small rise in the correction 
at $2m_W$ due to resonant box diagrams in that region. 
This feature is also present in the electroweak corrections 
to diphoton production in this observable \cite{Chiesa:2017gqx}. 
At large transverse momentum the usual Sudakov logarithms 
are recovered, resulting in corrections of around 
$-8\%$ at 1\,TeV for all photon-pair invariant masses. 
Most importantly, beyond $m_{\gamma\gamma}\gtrsim 300\,\text{GeV}$, 
the electroweak corrections are again not covered by the LO 
uncertainty estimate.

\begin{figure}[t!]
  \centering
  \includegraphics[width=0.32\textwidth]{figs_aaa/dphi_y1y2}
  \includegraphics[width=0.32\textwidth]{figs_aaa/dphi_y1y3}
  \includegraphics[width=0.32\textwidth]{figs_aaa/dphi_y2y3}
  \caption{
    Azimuthal separation of the leading and subleading photon (left),
    leading and third leading photon (centre), subleading and third leading 
    photon (right) 
    in triple photon production at the LHC at 13\,TeV. 
    Details as in Fig.\ \ref{fig:aaa:pt}.
    \label{fig:aaa:dphi}
  }
\end{figure}

Finally, in Figure \ref{fig:aaa:dphi} we show the azimuthal 
separation $\Delta\phi$ between all three diphoton pairs. 
Similar features as before are visible as both the NLO QCD 
and NLO EW corrections relax the kinematic boundaries of 
the leading order calculation. 
Especially the azimuthal separation of the leading and 
third leading photon receives substantial shape corrections 
throughout the entire spectrum, with and without the 
presence of a jet veto. The angular distribution between leading photon
and the second or third subleading photon exhibit a kinematical edge at $\pi/2$ which 
is relaxed at NLO. 
This kinematical edge can be understood in the following way:
 Let us consider the first observable, the angular separation between 
the leading and the subleading photon, the argument also holds for the separation between
the leading and the third leading photon. The third leading photon has to recoil against the
system of the two leading photons. Let us go into the limit where the three photons have
a very similar transverse momentum and consider the configuration where the leading
photon is back-to-back to the third leading photon. Then the second photon must 
be perpendicular to this axis. If the third photon has less transverse momentum, it 
needs the second photon to recoil against the leading photon. Therefore it gets closer
to the third photon and further away from the leading photon. $\pi/2$ is therefore the
minimal distance the second photon can have to the leading one.  Only at NLO where
one can have additional radiation this constraint is relaxed. We will see later in Fig.
\ref{fig:aaz:dphi} that this situation is also present when one replaces one of the subleading
photons by a $z$ boson. 
The electroweak corrections are negligible for this 
observable.


\subsection[\texorpdfstring{$\gamma\gamma\ell\nu$}{aalnu} production]
           {$\boldsymbol{\gamma\gamma\ell\nu}$ production}
\label{sec:results:aaw}

Next we move on to diphoton production in association with 
a $W$ boson decaying leptonically.
In this context, we will representatively focus on $W^-$ bosons 
as we do not expect qualitatively different results for $W^+$ 
bosons. 
We define our fiducial phase space by the following. 
First, we require the presence of exactly one charged lepton, 
dressed with all photons in a cone of size $R=0.1$, with 
$\pT>20\,\text{GeV}$ and $|\eta|<2.5$. 
Among the remaining photons we require at least two identified 
ones, using the procedure described in Sec.\ \ref{sec:results:aaa}, 
only changing the transverse momentum requirements to 
$\pT>20\,\text{GeV}$ for both the leading and the subleading 
identified photon. 
We further demand the angular separation of both photons 
to be $\Delta R(\gamma_1,\gamma_2)>0.4$ and each photon and 
the lepton to be $\Delta R(\ell,\gamma)>0.7$.
Furthermore we require the transverse mass of the lepton-neutrino system
to be larger than $40\, \text{GeV}$.
The inclusive cross sections and correction factors were 
detailed in Tab.\ \ref{tab:xsec}.


\begin{figure}[t!]
  \centering
  \includegraphics[width=0.32\textwidth]{figs_aaw/pT_y1}
  \includegraphics[width=0.32\textwidth]{figs_aaw/pT_y2}
  \includegraphics[width=0.32\textwidth]{figs_aaw/pT_l1}
  \caption{
    Transverse momentum of the leading (left), subleading (centre) 
    and third leading (right) photon at the LHC at 13\,TeV.
    \label{fig:aaw:pt}
  }
\end{figure}

Figure \ref{fig:aaw:pt} shows the transverse momenta of 
the leading and subleading photon as well as the charged 
lepton. 
As was observed in triple photon production, the inclusive 
NLO QCD corrections are large. 
Nonetheless, they differ substantially between the three 
oberservables and in their different regions. 
While $\deltaQCD$ rises from values of 0.3 at small 
transverse momenta of the leading photon to 2 at 
$\pT\approx 100\,\text{GeV}$, it remains constant above that 
value.
In the subleading photon's transverse momentum, the 
situation is different. 
Experiencing an acceptance cut induced jump from 0.3 to 2.5 at low \pT, 
it decreases steadily thereafter to reach $\deltaQCD=0.8$ 
at 500\,GeV. 
The behaviour of the QCD corrections to the lepton \pT\ are 
then again qualitatively similar to that of the leading 
photon, starting at $\deltaQCD=0.8$ at low \pT \, and rising 
steeply to 2.5 at 100\,GeV and then gradually leveling out 
at 3.5. 
Again, the presence of a restrictive jet veto has little 
effect on the small transverse momentum regions, but 
effectively contains the size of the NLO QCD corrections 
to remain below $\deltaQCD=0.5$. 
Instead, they are again driven negative, reaching 
$-20-30\%$ for all three observables. 

The electroweak corrections are negative throughout, 
reaching $-20\%$ for the leading photon and the lepton 
and $-25\%$ for the subleading photon at 500\,GeV. 
For this process, starting at transverse momenta of 
around 100\,GeV, the electroweak corrections are much 
larger than the LO uncertainty estimate.
Due to their larger size, as compared with the \aaa\ 
production process, the difference between their 
additive and multiplicative combination with the 
QCD corrections becomes more pronounced when no 
jet veto is applied. 
One reason is that the QCD corrections are dominated 
by real emission contributions, which only receive the 
not-so-small electroweak correction factor in the 
multiplicative combination, but not the additive one. 
While this correction factor is, strictly speaking, 
associated only with the Born configuration, its 
application to real-emission configurations is well 
motivated in the electroweak Sudakov-regime at 
large transverse momenta. 
The concurrence of the additive and multiplicative 
schemes in the presence of the jet veto now precisely 
originates in the taming of the QCD corrections by 
restricting the influence of real-emission topologies. 

\begin{figure}[t!]
  \centering
  \includegraphics[width=0.32\textwidth]{figs_aaw/m_y1y2_comb_log}
  \includegraphics[width=0.32\textwidth]{figs_aaw/m_y1l1_comb_log}
  \includegraphics[width=0.32\textwidth]{figs_aaw/m_t_comb_log}
  \caption{
    Pairwise invariant mass of the leading and subleading photon (left),
    leading and third leading photon (centre), subleading and third leading 
    photon (right) at the LHC at 13\,TeV.
    \label{fig:aaw:myy}
  }
\end{figure}

Considering now the invariant mass of both photons 
and the leading photon and the lepton, presented in 
Figure \ref{fig:aaw:myy}, a similar picture 
presents itself. 
Both QCD corrections are moderate at small invariant 
masses and increase as the invariant mass rises. 
While $m_{\gamma_1\gamma_2}$ developes a correction 
factor of $\deltaQCD=2$ at 200\,GeV and then falls 
again to 0.4 at 1\,TeV, $m_{\gamma_1\ell}$ only 
reaches a maximum of $\deltaQCD=1.3$ before falling 
to 0.4 as well. 
In the presence of a jet veto, the shape of the corrections 
remains similar, but their magnitude is contained to 
be smaller than 0.5, turning negative around 500\,GeV 
for both observables. 
The electroweak corrections are again negative 
throughout, reaching $-15-20\%$ at invariant masses 
of 1\,TeV.
Again, above 200\,GeV they are consistently outside 
the LO uncertainty estimate and play an important 
role.

Figure \ref{fig:aaw:myy} also displays the transverse 
mass of the $W$ boson on the right hand side. 
This distribution exhibits the general features known 
from inclusive $W$ production, only the QCD corrections 
are scaled to the values present in this process.
Electroweak corrections are less pronounced than for 
the invariant masses, but are still non-negligible. 
They reach about $-8\%$ at $m_\text{T}=200\,\text{GeV}$ 
and $-14\%$ at 1\,TeV.

\begin{figure}[t!]
  \centering
  \includegraphics[width=0.32\textwidth]{figs_aaw/dphi_y1_y2}
  \includegraphics[width=0.32\textwidth]{figs_aaw/dphi_y1_l1}
  \includegraphics[width=0.32\textwidth]{figs_aaw/dphi_y2_l1}
  \caption{
    Azimuthal separation of the leading and subleading photon (left),
    leading and third leading photon (centre), subleading and third leading 
    photon (right) at the LHC at 13\,TeV.
    \label{fig:aaw:dphi}
  }
\end{figure}

Finally, Figure \ref{fig:aaw:dphi} presents the azimuthal 
separations of the leading and the subleading photon, and 
both the leading and subleading photon and the lepton. 
Contrary to \aaa\ production, no kinematic constraints 
are present at LO, owing to the difference in flavour 
between the considered objects and the fact that the 
lepton is only one of two decay products of the $W$ which 
takes the role of the third photon from point of view of 
the contributing diagrams. 

Both the QCD and EW corrections are generally flat, taking 
values of typically $\deltaQCD=1-2$ in absence and 
$\deltaQCD=0.4$ in presence of a jet veto, and 
$\deltaEW=-2\%$.


As none of the observables considered for this process 
exhibits a kinematic boundary at LO that is lifted at 
NLO, both the additive and the multiplicative combination 
of QCD and EW corrections present a viable prediction 
throughout and their difference can be taken as an 
indication of the potential size of higher-order corrections 
of $\order(\alphas\alpha)$.



\subsection[\texorpdfstring{$\gamma\gamma\ell^+\ell^-$}{aall} production]
           {$\boldsymbol{\gamma\gamma\ell^+\ell^-}$ production}
\label{sec:results:aaz}

Finally, we consider diphoton production in association with 
a $Z$ boson in its leptonic decay channel. 
In practise we look at lepton pair production including 
all non-resonant diagrams.
The fiducial phase space of our analysis is defined as follows. 
First, we require the presence of exactly one lepton pair 
of opposite charge, dressed with all photons in a cone of 
size $R=0.1$, with $\pT>20\,\text{GeV}$ and $|\eta|<2.47$. 
Their invariant mass must be larger than 40\,GeV. 
As in the case of \aaw production, we then require at least 
two identified photons among the remaining ones after dressing, 
using the procedure described in Sec.\ \ref{sec:results:aaa}, 
loosening the transverse momentum requirements to 
$\pT>15\,\text{GeV}$ for both the leading and the subleading 
identified photon. 
We further demand the angular separation of both photons 
to be $\Delta R(\gamma_1,\gamma_2)>0.4$ and each photon and 
the lepton to be $\Delta R(\ell,\gamma)>0.4$.
The inclusive cross sections and correction factors were 
detailed in Tab.\ \ref{tab:xsec}.

\begin{figure}[t!]
  \centering
  \includegraphics[width=0.32\textwidth]{figs_aaz/pT_y1}
  \includegraphics[width=0.32\textwidth]{figs_aaz/pT_y2}
  \includegraphics[width=0.32\textwidth]{figs_aaz/pT_l1l2_comb_log}
  \caption{
    Transverse momentum of the leading (left) and subleading (centre) 
    photon as well as the dressed lepton pair (right) 
    in diple photon production in association with a lepton pair 
    at the LHC at 13\,TeV. 
    Details as in Fig.\ \ref{fig:aaa:pt}.
    \label{fig:aaz:pt}
  }
\end{figure}

In close analogy with the previous two processes considered, 
we start our discussion of the results with the transverse 
momenta of the leading and subleading photon and the charged 
lepton pair, cf.\ Figure \ref{fig:aaz:pt}. 
The first observation is, while the general behaviour of the 
spectra is the same as in \aaw production, the inclusive QCD 
corrections are much smaller in this case. 
They range between $\deltaQCD=0.3-0.4$ for the leading and 
subleading photon and rise to 0.7 for the lepton pair at 
500\,GeV. 
The restrictive jet veto employed for the previous process 
reduces the QCD corrections further. 
Beyond the negligible impact at very small transverse 
momenta, they turn negative early on, reaching $\deltaQCD=-0.2-0.4$ 
at 500\,GeV. 
The electroweak corrections are negative throughout, increasing 
from $\deltaEW=-4\%$ at low transverse momenta to $-20\%$ at 
500\,GeV.
They are consistently larger than the LO uncertainty estimate. 
As a consequence, the difference between the additive and 
multiplicative is comparably large, especially in the high 
\pT\ regions. 

\begin{figure}[t!]
  \centering
  \includegraphics[width=0.32\textwidth]{figs_aaz/m_y1y2_comb_log}
  \includegraphics[width=0.32\textwidth]{figs_aaz/m_y1l1l2_comb_log}
  \includegraphics[width=0.32\textwidth]{figs_aaz/m_y2l1l2_comb_log}
  \caption{
    Pairwise invariant mass of the leading and subleading photon (left),
    the leading photon and the lepton pair (centre), and the subleading 
    photon and the lepton pair (right)
    in diple photon production in association with a lepton pair 
    at the LHC at 13\,TeV. 
    Details as in Fig.\ \ref{fig:aaa:pt}.
    \label{fig:aaz:myy}
  }
\end{figure}

Figure \ref{fig:aaz:myy} continues to display the invariant masses 
of the leading and subleading photon, and both the leading and 
subleading photon and the lepton pair. 
The familiar peak in $m_{\gamma_i\ell^+\ell^-}$ is produced by 
configurations where the respective photon is emitted by the 
lepton-pair from a resonant $Z$ boson decay. 
Again, the QCD corrections are moderate, always staying below 
$\deltaQCD=0.4$ and of a similar shape as encoutered already in 
\aaw\ production. 
The jet veto again reduces the QCD corrections and drives them negative 
in large portions of the observable range, reaching as much as 
$\deltaQCD=-0.5$ for the diphoton invariant mass. 
The electroweak corrections, on the other hand, are qualitatively very similar 
to the \aaw case: with the exception of the well known positive 
radiative corrections below the $Z$ peak, they are negative 
throughout, reaching $\deltaEW=-20\%$ at invariant masses of 
1\,TeV.
As in the case of the transverse momenta, due to the similar size 
of the QCD and electroweak corrections at large transverse 
momenta, the difference between their additive and the multiplicative 
combination is amplified.

\begin{figure}[t!]
  \centering
  \includegraphics[width=0.32\textwidth]{figs_aaz/dphi_y1_y2}
  \includegraphics[width=0.32\textwidth]{figs_aaz/dphi_y1_l1l2}
  \includegraphics[width=0.32\textwidth]{figs_aaz/dphi_y2_l1l2}
  \caption{
    Azimuthal separation of the leading and subleading photon (left),
    the leading photon and the lepton pair (centre), and the subleading 
    photon and the lepton pair (right)
    in diple photon production in association with a lepton pair 
    at the LHC at 13\,TeV. 
    Details as in Fig.\ \ref{fig:aaa:pt}.
    \label{fig:aaz:dphi}
  }
\end{figure}

In Figure \ref{fig:aaz:dphi} we consider the azimuthal separation 
between the leading and subleading photon, and both the leading 
and subleading photon and the lepton pair.
While both photons tend to be back-to-back, the azimuthal 
separation of the leading photon and the lepton pair exhibits 
a kinematic edge at LO, restricting it to be larger than 
$\tfrac{1}{2}\,\pi$. 
As discussed above this kinematic edge is also present in the case of the triple photon process
(see Fig. \ref{fig:aaa:dphi_}) when considering the azimuthal angle between
the leading and any subleading photon.
On the contrary, the azimuthal separation of the subleading 
photon and the lepton pair exhibits no such limit at LO and 
instead peaks at roughly $\Delta\phi=\tfrac{1}{2}\,\pi$. 
The QCD corrections, with the exception of the aforementioned 
edge, are generally flat, taking values of $\deltaQCD\approx 0.3$ 
in the absence and $\deltaQCD\approx 0.2$ in the presence 
of a jet veto. 
The electroweak corrections are equally flat and amount to 
$\deltaEW\approx-4\%$. 
Their additive and multiplicative combinations are very 
similar.

\begin{figure}[t!]
  \setlength{\unitlength}{\textwidth}
  \begin{picture}(0,0.37)
    \put(0,0.24){\includegraphics[width=0.32\textwidth]{figs_aaz_aa-ind/pT_y1}}
    \put(0,0.12){\includegraphics[width=0.32\textwidth]{figs_aaz_aa-ind/m_y1y2_comb_log}}
    \put(0,0){\includegraphics[width=0.32\textwidth]{figs_aaz_aa-ind/dphi_y1_y2}}
    \put(0.33,0.24){\includegraphics[width=0.32\textwidth]{figs_aaz_aa-ind/pT_y2}}
    \put(0.33,0.12){\includegraphics[width=0.32\textwidth]{figs_aaz_aa-ind/m_y1l1l2_comb_log}}
    \put(0.33,0){\includegraphics[width=0.32\textwidth]{figs_aaz_aa-ind/dphi_y1_l1l2}}
    \put(0.66,0.24){\includegraphics[width=0.32\textwidth]{figs_aaz_aa-ind/pT_l1l2_comb_log}}
    \put(0.66,0.12){\includegraphics[width=0.32\textwidth]{figs_aaz_aa-ind/m_y2l1l2_comb_log}}
    \put(0.66,0){\includegraphics[width=0.32\textwidth]{figs_aaz_aa-ind/dphi_y2_l1l2}}
  \end{picture}
  \caption{
    Contribution of $\gamma\gamma$-induced production channels at LO.
    \label{fig:aaz:aa-ind}
  }
\end{figure}

Finally, \aaz\ production is the only process of the three processes 
considered in this publication which can be produced through 
photon-induced channels at leading order, 
$\gamma\gamma\to\ell^+\ell^-\gamma\gamma$. 
To this end, we investigate their contribution at LO in 
Figure \ref{fig:aaz:aa-ind}. 
The photon-induced contribution is typically a few per mille of the 
LO cross section, but raises up to 2\% for invariant masses larger 
than 1\,TeV, especially $m_{\gamma_i\ell^+\ell^-}$. 
In this region, however, typical QCD and EW corrections are much larger 
such that this contribution can be safely ignored. 
As the photon-induced processes only contribute through non-resonant 
lepton-pair production, their contribution can be further suppressed 
by tightening the acceptance window in the lepton-pair invariant mass.


