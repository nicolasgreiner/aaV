\section{Results}
\label{sec:results}

In this section we present numerical results for the NLO QCD and NLO EW 
corrections to all three production processes of a diphoton pair in 
association with a third vector boson, a third photon, a $W$ or a $Z$ 
boson, at the LHC at a centre-of-mass energy of 13\,TeV. 
In case of an accompanying $W$ or a $Z$ boson, we consider the full 
off-shell leptonic final state, i.e.\ lepton-neutrino or lepton-pair 
production.
All results are obtained in the Standard Model using the complex-mass 
scheme \cite{Denner:2005fg} with the following input parameters
\begin{center}
  \begin{tabular}{rclrcl}
    $\alphazero$ &\shortequal& $1/137.03599976$  \qquad &&& \\
    $\Gmu$ &\shortequal& $1.1663787\times 10^{-5}\; \text{GeV}^2$ &&& \\
    $m_W$ &\shortequal& $80.385\; \text{GeV}$       & $\Gamma_W$ &\shortequal& $2.085\; \text{GeV}$ \\
    $m_Z$ &\shortequal& $91.1876\; \text{GeV}$      & $\Gamma_Z$ &\shortequal& $2.4952\; \text{GeV}$ \\
    $m_h$ &\shortequal& $125.0\; \text{GeV}$        & $\Gamma_h$ &\shortequal& $0$\\
    $m_t$ &\shortequal& $173.2\; \text{GeV}$        & $\Gamma_t$ &\shortequal& $0$\;.
  \end{tabular}
\end{center}
While we calculate triple photon production in the \alphazero-scheme, 
we use a mixed scheme for \aaw\ and \aaz production: at LO two 
powers of $\alpha$ are taken in the \alphazero-scheme, while 
one power of $\alpha$ is taken in the \Gmu-scheme. 
The additional power of $\alpha$ in the NLO EW correction 
is evaluated in \alphazero-scheme again.
The virtual amplitudes are renormalised correspondingly.
In all cases both the width of the top quark and the Higgs boson 
can safely be neglected as there are no diagrams containing either 
as $s$-channel propagators which can potentially go on-shell. 
All other lepton and parton masses and widths are set to zero, 
i.e.\ we are working in the five-flavour scheme.
We use the \textsc{CT14nlo} PDF set with $\alphas(m_Z)=0.118$ 
\cite{}. 
The use of a QCD-only PDF is justified by the fact that, 
at LO, the photon induced corrections are either non-existant 
(\aaa, \aaw) or negligible (\aaz).
This finding will be detailed in Section \ref{sec:results:aaz}.

We define our central scales through
\begin{equation}
  \label{eq:murfdef}
  \begin{split}
    \muR^0 = \muF^0 = \left\{
    \begin{array}{ll}
      m_{\aaa} \qquad & \text{in \aaa\ production} \\[2mm]
      \tfrac{1}{2}\,H_T' & \text{in \aaw\ and \aaz\ production.}
    \end{array}\right.
  \end{split}
\end{equation}
Therein, the modified scalar sum of transverse momenta is defined as 
\begin{equation}
  \label{ew:defHT}
  \begin{split}
    H_\mathrm{T}' = E_\mathrm{T}^V + \sum_{\gamma,q,g} p_{\mathrm{T},i}
  \end{split}
\end{equation}
with $\left.E_\mathrm{T}^W\right.^2=(p_\ell+p_\nu)^2$ and 
$\left.E_\mathrm{T}^Z\right.^2=(p_{\ell^+}+p_{\ell^-})^2$ 
in full analogy to the case of the case of vector boson production 
in association with jets \cite{}.
As the Born process in each case has no $\muR$ dependence, we 
do not expect the choice of scale to have a significant influence 
on the size of the relative QCD and EW corrections.
We calculate the leading order cross section $\rd\sigma_\text{LO}(\muF)$, 
which only depends on the factorisation scale $\muF$, the 
NLO QCD differential correction factor $\deltaQCD(\muR,\muF)$, 
introducing the additional $\muR$-dependence, and the NLO EW 
differential correction factor $\deltaEW(\muF^0)$. 
To estimate the impact of yet to-be-calculated higher-order 
corrections we vary the free scales $\muR$ and $\muF$ 
by the conventional factor of two around their central values 
$\muR^0$ and $\muF^0$, respectively.
We do not vary the factorisation scale for the determination 
of $\deltaEW$ as the inherent, albeit normally phenomenologically 
irrelevant, stabilisation of the $\muF$-dependence at NLO EW 
is not reflected in our chosen PDF.
Hence, our NLO EW result exhibits the exact same $\muF$-dependence 
as the LO result.
We define 
\begin{equation}
  \label{eq:defnlo}
  \begin{split}
    \rd\sigma_\text{NLO QCD}
    \,=\;&\rd\sigma_\text{LO}\left(1+\deltaQCD\right)\\
    \rd\sigma_\text{NLO EW}
    \,=\;&\rd\sigma_\text{LO}\left(1+\deltaEW\right)\\
    \rd\sigma_\text{NLO \QCDpEW}
    \,=\;&\rd\sigma_\text{LO}\left(1+\deltaQCD+\deltaEW\right)\\
    \rd\sigma_\text{NLO \QCDtEW}
    \,=\;&\rd\sigma_\text{LO}\left(1+\deltaQCD\right)\left(1+\deltaEW\right)\;.
  \end{split}
\end{equation}
Therein, the difference between NLO \QCDpEW\ and NLO \QCDtEW, which 
is of relative $\order(\alphas\alpha)$, can serve as an indicator 
of the potential size of unknown corrections at that order.

In Table \ref{tab:xsec} we quote the inclusive cross sections for all 
three processes. 
The fiducial cuts for each process is detailed in its 
respective subsection below. 
We note that the NLO QCD corrections for both triple photon 
production and \aaw\ production are strongly jet veto dependent, 
a result that was previously discussed in great detail in 
\cite{Bozzi:2011en,Bozzi:2011wwa}, and will be revisited in the 
following. 
A much milder jet veto dependence is found for \aaz\ production. 
The electroweak corrections to inclusive cross sections are 
generally much smaller, ranging from ..\% (\aaa) to ..\% (\aaw) 
and ..\% (\aaz).

\begin{table}[t!]
  \centering
  \begin{tabular}{l||c|c|c}
      & $\;\;pp \to \gamma \gamma\gamma\;\;$
      & $\;\;pp \to \gamma \gamma e^-\bar\nu_e\;\;$ 
      & $\;\;pp \to \gamma \gamma e^+e^-\;\;$ \\
    \hline\hline
    $\sigma_\text{LO}\;\;[\text{pb}] $ &  &  & \\
    \hline
    $\delta_\text{QCD}\;\;[\text{pb}]\;\;\pTveto=\infty $ &  &  & \\
    \hline
    $\delta_\text{QCD}\;\;[\text{pb}]\;\;\pTveto=30\,\text{GeV} $ &  &  & \\
    \hline
    $\delta_\text{EW}\;\;[\text{pb}] $ &  &  & \\
  \end{tabular}
  \caption{
    Total cross sections at LO, NLO QCD and NLO EW for $\gamma\gamma\gamma$, 
    $\gamma\gamma e^-\bar\nu_e$ and $\gamma\gamma e^+e^-$
    production at 13\,TeV at the LHC.
    \label{tab:xsec}
  } 
\end{table}


\subsection[\texorpdfstring{$\gamma\gamma\gamma$}{aaa} production]
           {$\boldsymbol{\gamma\gamma\gamma}$ production}
\label{sec:results:aaa}

The triple photon production process is defined 
by the presence of three identified photons in the 
central detector. 
To this end we use the smooth cone isolation 
criterion \cite{Frixione:1998jh}, limiting the amount of 
hadronic activity in a cone $R_\gamma$ to
\begin{equation}
  \begin{split}
    E_{{\rm had, max}} (r_{\gamma})
    = \epsilon\, \pT^{\gamma} \left( \frac{1-\cos r_\gamma}
				       {1-\cos R_\gamma}\right)^{n}\;,
  \end{split}
  \label{eq:frix}
\end{equation}
where $r_{\gamma}$ denotes the angular separation between the photon and 
the parton, with 
\begin{equation}
  \label{eq:coneparams}
  \begin{split}
    R_{\gamma}=0.4\;, \quad \epsilon = 0.05\;, \quad n = 1\;, 
  \end{split}
\end{equation}
to define isolated photon candidates.
These candidates are then ordered in transverse momentum. 
We then require at least three such candidates within $|\eta|<2.37$, 
the leading one of which needs $\pT>40\,\text{GeV}$, 
while the all subleading ones need only 
$\pT>30\,\text{GeV}$. 
Finally, a pairwise separation of $\Delta R(\gamma_i,\gamma_j)>0.4$ 
between all identified photons is required.
It is worth noting that at NLO EW it is possible to find more 
than three isolated photons, in which case any combination may 
fulfill the above criteria. 

\begin{figure}[t!]
  \centering
  \includegraphics[width=0.32\textwidth]{figs_aaa/pT_y1}
  \includegraphics[width=0.32\textwidth]{figs_aaa/pT_y2}
  \includegraphics[width=0.32\textwidth]{figs_aaa/pT_y3}
  \caption{
    Transverse momentum of the leading (left), subleading (centre) 
    and third leading (right) photon 
    in triple photon production at the LHC at 13\,TeV. 
    The distributions are shown at LO (blue), NLO QCD (green), 
    NLO EW (orange), NLO \QCDpEW\ (red) and NLO \QCDtEW\ (black) 
    including scale uncertainties. The top ratio plot details 
    the relative corrections to the leading order cross section 
    without applying any jet veto, while the centre ratio plot 
    applies a jet veto of $p_\mathrm{T,jet}^\mathrm{veto}=\text{30\,GeV}$. 
    The lower ratio highlights the size of the electroweak corrections.
    \label{fig:aaa:pt}
  }
\end{figure}

Figure \ref{fig:aaa:pt} displays the transverse momenta of the 
first three leading photons. 
The NLO QCD corrections exhibit a handful of interesting features. 
Firstly, in the absence of a jet veto the fixed-order calculation 
exhibits a huge correction factors, mainly induced by the 
opening of new channels in the real emission. 
Similar, kinematic constraints present at LO \footnote{
  At leading order the leading photon needs to be in a different 
  hemisphere than both the subleading and third leading photon. 
  Thus, $\Delta\phi_{\gamma_1\gamma_2}$ and $\Delta\phi_{\gamma_1\gamma_3}$ 
  must be larger that $\tfrac{1}{2}\,\pi$.
} are released and 
lead to a larger phase space that can be populated. 
These findings mandate the inclusion of at least the $\aaa+\text{jet}$ 
production process at NLO QCD to arrive at a reliable description 
of inclusive \aaa\ production, and thus either a NNLO QCD calculation 
or a multijet merging ansatz \cite{Hoeche:2012yf,Kallweit:2015dum}. 
While the inclusive QCD corrections at very small transverse momenta 
are universally large for all three leading photon \pT-spectra 
($\deltaQCD\approx 2$), they remain at exorbitantly large 
($\deltaQCD\approx 1.5$) throughout the considered range 
only for the leading jet \pT. 
For both subleading photons the QCD corrections are quickly 
decreasing, leveling out at a $\approx 20\%$ at large transverse 
momenta.
In the presence of a restrictive jet veto the very low transverse 
momentum region still experiences large correction of about 
$\deltaQCD\approx 1$. 
As transverse momenta are increasing, the QCD corrections now turn 
negative reaching now $-50-60\%$ and for all three photons. 

The electroweak corrections, due to the absence of the opening 
of large new channels at the next-to-leading order, are dominated 
by the virtual corrections. 
Consequently, the phase space restrictions present at leading 
order are still respected.
For all three photons they are small but positive at small 
transverse momenta and exhibit the usual Sudakov shape at 
large transverse momenta.
They reach $-10\%$ for the leading and subleading photon and 
$-20\%$ for the third photon at $\pT=500\,\text{GeV}$. 
Due to the different sizes of the QCD and electroweak 
corrections, the additive and multiplicative combination 
of corrections lead to very similar results.

\begin{figure}[t!]
  \centering
  \includegraphics[width=0.32\textwidth]{figs_aaa/m_y1y2_comb_log}
  \includegraphics[width=0.32\textwidth]{figs_aaa/m_y1y3_comb_log}
  \includegraphics[width=0.32\textwidth]{figs_aaa/m_y2y3_comb_log}
  \caption{
    Pairwise invariant mass of the leading and subleading photon (left),
    leading and third leading photon (centre), subleading and third leading 
    photon (right) 
    in triple photon production at the LHC at 13\,TeV. 
    Details as in Fig.\ \ref{fig:aaa:pt}.
    \label{fig:aaa:myy}
  }
\end{figure}

Figure \ref{fig:aaa:myy} continues with the three diphoton 
invariant masses. 
The \pT\ and $\Delta R$ requirements 
of the event selection induce a minimum in the distributions 
at leading order. 
The region below can only be filled if a fourth particle is 
present, as is the case in both the QCD and electroweak real 
emission corrections, leading to simultaneously huge 
corrections \deltaQCD\ and \deltaEW\ as the Born cross 
section vanishes. 
Due to this behaviour, the multiplicative combination of 
corrections, NLO \QCDtEW, ceases to be well defined and 
spikes in the distribution are visible.
The distributions below, where $\rd\sigma_\text{LO}=0$, 
are ill-defined. 
In consequence, for distributions where kinematic boundaries 
exist at leading order, but are lifted at higher orders, 
the multiplicative combination does not present a viable 
option for a good description of the observable 
throughout phase space. 

The QCD corrections themselves again exceed $200\%$ at 
small invariant masses, already well before the above 
described kinematic boundary effect takes hold. 
As the invariant masses are increasing, the 
QCD corrections are dropping to $20-30\%$ for the inclusive 
selection. 
The structure the QCD corrections exhibit around 80-90\,GeV 
in all three diphoton-pair invariant masses are induced by 
the acceptance cuts.
In the presence of the jet veto, the QCD corrections are 
reduced and turn negative beyond $m_{\gamma\gamma}\gtrsim 300\,\text{GeV}$ 
reaching around $-40\%$ at 1\,TeV. 
The electroweak corrections, on the other hand, are 
again moderate, ranging from $+1.5\%$ between 70 and 200\,GeV 
for $m_{\gamma_1\gamma_2}$ and $m_{\gamma_1\gamma_2}$ and 
0 and 100\,GeV for $m_{\gamma_2\gamma_3}$. 
$m_{\gamma_1\gamma_2}$ exhibits a small rise in the correction 
at $2m_W$ due to resonant box diagrams in that region. 
This feature is also present in the electroweak corrections 
to diphoton production in this observable \cite{Chiesa:2017gqx}. 
At large transverse momentum the usual Sudakov logarithms 
are recoveren, resulting in corrections of around 
$-8\%$ at 1\,TeV for all photon-pair invariant masses. 

\begin{figure}[t!]
  \centering
  \includegraphics[width=0.32\textwidth]{figs_aaa/dphi_y1y2}
  \includegraphics[width=0.32\textwidth]{figs_aaa/dphi_y1y3}
  \includegraphics[width=0.32\textwidth]{figs_aaa/dphi_y2y3}
  \caption{
    Azimuthal separation of the leading and subleading photon (left),
    leading and third leading photon (centre), subleading and third leading 
    photon (right) 
    in triple photon production at the LHC at 13\,TeV. 
    Details as in Fig.\ \ref{fig:aaa:pt}.
    \label{fig:aaa:dphi}
  }
\end{figure}

Finally, in Figure \ref{fig:aaa:dphi} we show the azimuthal 
separation $\Delta\phi$ between all three diphoton pairs. 
Similar features as before are visible as both the NLO QCD 
and NLO EW corrections relax the kinematic boundaries of 
the leading order calculation. 
Especially the azimuthal separation of the leading and 
third leading photon receives substantial shape corrections 
throughout the entire spectrum, with and without the 
presence of a jet veto. 
The electroweak corrections are negligible for this 
observable.


\begin{itemize}
  \item \pT of 3rd photon much softer than 1st and 2nd
  \item in all cases NLO EW outside LO uncertainty band
\end{itemize}



\subsection[\texorpdfstring{$\gamma\gamma\ell\nu$}{aalnu} production]
           {$\boldsymbol{\gamma\gamma\ell\nu}$ production}
\label{sec:results:aaw}

\comment{MS: is that $\ell^+$ or $\ell^-$?}

\begin{figure}[t!]
  \centering
  \includegraphics[width=0.32\textwidth]{figs_aaw/pT_y1}
  \includegraphics[width=0.32\textwidth]{figs_aaw/pT_y2}
  \includegraphics[width=0.32\textwidth]{figs_aaw/pT_l1}
  \caption{
    Transverse momentum of the leading (left), subleading (centre) 
    and third leading (right) photon at the LHC at 13\,TeV.
    \label{fig:aaw:pt}
  }
\end{figure}

\begin{figure}[t!]
  \centering
  \includegraphics[width=0.32\textwidth]{figs_aaw/m_y1y2_comb_log}
  \includegraphics[width=0.32\textwidth]{figs_aaw/m_y1l1_comb_log}
  \includegraphics[width=0.32\textwidth]{figs_aaw/m_t_comb_log}
  \caption{
    Pairwise invariant mass of the leading and subleading photon (left),
    leading and third leading photon (centre), subleading and third leading 
    photon (right) at the LHC at 13\,TeV.\\
    \comment{MS: Spike in $m_{\gamma_1\gamma_2}$ is in NLO \QCDtEW 
             and comes from $\deltaEW\approx 10$ in this bin at the 
             edge of the LO phase space.}
    \label{fig:aaw:myy}
  }
\end{figure}

\begin{figure}[t!]
  \centering
  \includegraphics[width=0.32\textwidth]{figs_aaw/dphi_y1_y2}
  \includegraphics[width=0.32\textwidth]{figs_aaw/dphi_y1_l1}
  \includegraphics[width=0.32\textwidth]{figs_aaw/dphi_y2_l1}
  \caption{
    Azimuthal separation of the leading and subleading photon (left),
    leading and third leading photon (centre), subleading and third leading 
    photon (right) at the LHC at 13\,TeV.\\
    \comment{MS: Spike in $m_{\gamma_1\gamma_2}$ is in NLO \QCDtEW 
             and comes from $\deltaEW\approx 10$ in this bin at the 
             edge of the LO phase space.}
    \label{fig:aaw:dphi}
  }
\end{figure}


\subsection[\texorpdfstring{$\gamma\gamma\ell^+\ell^-$}{aall} production]
           {$\boldsymbol{\gamma\gamma\ell^+\ell^-}$ production}
\label{sec:results:aaz}

\begin{figure}[t!]
  \centering
  \includegraphics[width=0.32\textwidth]{figs_aaz/pT_y1}
  \includegraphics[width=0.32\textwidth]{figs_aaz/pT_y2}
  \includegraphics[width=0.32\textwidth]{figs_aaz/pT_l1l2_comb_log}
  \caption{
    Transverse momentum of the leading (left) and subleading (centre) 
    photon as well as the dressed lepton pair (right) 
    in diple photon production in association with a lepton pair 
    at the LHC at 13\,TeV. 
    Details as in Fig.\ \ref{fig:aaa:pt}.
    \label{fig:aaz:pt}
  }
\end{figure}

\begin{figure}[t!]
  \centering
  \includegraphics[width=0.32\textwidth]{figs_aaz/m_y1y2_comb_log}
  \includegraphics[width=0.32\textwidth]{figs_aaz/m_y1l1l2_comb_log}
  \includegraphics[width=0.32\textwidth]{figs_aaz/m_y2l1l2_comb_log}
  \caption{
    Pairwise invariant mass of the leading and subleading photon (left),
    the leading photon and the lepton pair (centre), and the subleading 
    photon and the lepton pair (right)
    in diple photon production in association with a lepton pair 
    at the LHC at 13\,TeV. 
    Details as in Fig.\ \ref{fig:aaa:pt}.\\
    \comment{MS: Spike in $m_{\gamma_1\ell_1\ell_2}$ is in NLO \QCDtEW\ 
             and comes from $\deltaEW\approx 10$ in this bin at the 
             edge of the LO phase space.}
    \label{fig:aaz:myy}
  }
\end{figure}

\begin{figure}[t!]
  \centering
  \includegraphics[width=0.32\textwidth]{figs_aaz/dphi_y1_y2}
  \includegraphics[width=0.32\textwidth]{figs_aaz/dphi_y1_l1l2}
  \includegraphics[width=0.32\textwidth]{figs_aaz/dphi_y2_l1l2}
  \caption{
    Azimuthal separation of the leading and subleading photon (left),
    the leading photon and the lepton pair (centre), and the subleading 
    photon and the lepton pair (right)
    in diple photon production in association with a lepton pair 
    at the LHC at 13\,TeV. 
    Details as in Fig.\ \ref{fig:aaa:pt}.\\
    \comment{MS: Spike in $\Delta\phi_{\gamma_1,\ell_1\ell_2}$ is in NLO \QCDtEW\ 
             and comes from $\deltaEW\approx 10$ in this bin at the 
             edge of the LO phase space.}
    \label{fig:aaz:dphi}
  }
\end{figure}



\begin{figure}[t!]
  \setlength{\unitlength}{\textwidth}
  \begin{picture}(0,0.37)
    \put(0,0.24){\includegraphics[width=0.32\textwidth]{figs_aaz_aa-ind/pT_y1}}
    \put(0,0.12){\includegraphics[width=0.32\textwidth]{figs_aaz_aa-ind/pT_y2}}
    \put(0,0){\includegraphics[width=0.32\textwidth]{figs_aaz_aa-ind/pT_l1l2_comb_log}}
    \put(0.33,0.24){\includegraphics[width=0.32\textwidth]{figs_aaz_aa-ind/m_y1y2_comb_log}}
    \put(0.33,0.12){\includegraphics[width=0.32\textwidth]{figs_aaz_aa-ind/m_y1l1l2_comb_log}}
    \put(0.33,0){\includegraphics[width=0.32\textwidth]{figs_aaz_aa-ind/m_y2l1l2_comb_log}}
    \put(0.66,0.24){\includegraphics[width=0.32\textwidth]{figs_aaz_aa-ind/dphi_y1_y2}}
    \put(0.66,0.12){\includegraphics[width=0.32\textwidth]{figs_aaz_aa-ind/dphi_y1_l1l2}}
    \put(0.66,0){\includegraphics[width=0.32\textwidth]{figs_aaz_aa-ind/dphi_y2_l1l2}}
  \end{picture}
  \caption{
    Contribution of $\gamma\gamma$-induced production channels at LO.
    \label{fig:aaz:aa-ind}
  }
\end{figure}


