\documentclass[12pt]{article}

%\textwidth=20cm
%\textheight=16cm
%\setlength{\oddsidemargin}{-1cm}
\pdfoutput=1
% \setlength\parindent{0pt}

\usepackage{jheppub}
\usepackage{graphicx,epsfig,amsmath,amssymb,booktabs}
%\usepackage{graphicx,epsfig,amsmath,amssymb}
%\usepackage{multirow,bigdelim,lscape}
%\usepackage{caption}
\usepackage{caption,subcaption}
\usepackage{xspace}

%\usepackage{listings}
%\lstset{basicstyle=\ttfamily}
%% 
%\usepackage{titlesec}

\newlength{\unitcharwidth}
\settowidth{\unitcharwidth}{1}

\newlength{\unitsuperscriptwidth}
\settowidth{\unitsuperscriptwidth}{\scriptsize 1}


%%% Macros
\newcommand{\rd}{\ensuremath{\mathrm{d}}}
\newcommand{\eqn}[1]{Eq.~(\ref{#1})}
\newcommand{\Gmu}{\ensuremath{G_\mu}}
\newcommand{\alphazero}{\ensuremath{\alpha(0)}}
\newcommand{\alphaGmu}{\ensuremath{\alpha_{\Gmu}}}
\newcommand{\alphas}{\ensuremath{\alpha_s}}
\newcommand{\aaa}{\ensuremath{\gamma\gamma\gamma}}
\newcommand{\aaw}{\ensuremath{\gamma\gamma W}}
\newcommand{\aaz}{\ensuremath{\gamma\gamma Z}}
\newcommand{\shortequal}{\ensuremath{\!\!\!=\!\!\!}}
\newcommand{\pT}{\ensuremath{p_\text{T}}}
\newcommand{\pTveto}{\ensuremath{p_\text{T,jet}^\text{veto}}}
\newcommand{\order}{\ensuremath{\mathcal{O}}}
\newcommand{\vP}{\ensuremath{\vphantom{\int_a^b}}}

\newcommand{\Sherpa}{S\scalebox{0.8}{HERPA}\xspace}
\newcommand{\GoSam}{G\scalebox{0.8}{O}S\scalebox{0.8}{AM}\xspace}
\newcommand{\Recola}{R\scalebox{0.8}{ECOLA}\xspace}
\newcommand{\OpenLoops}{O\scalebox{0.8}{PEN}L\scalebox{0.8}{OOPS	}\xspace}
\newcommand{\QGraf}{QG\scalebox{0.8}{RAF}\xspace}
\newcommand{\FORM}{F\scalebox{0.8}{ORM}\xspace}
\newcommand{\Spinney}{S\scalebox{0.8}{PINNEY}\xspace}
\newcommand{\Ninja}{N\scalebox{0.8}{INJA}\xspace}
\newcommand{\Samurai}{S\scalebox{0.8}{AMURAI}\xspace}
\newcommand{\GolemNF}{G\scalebox{0.8}{OLEM}95\xspace}
\newcommand{\OneLoop}{O\scalebox{0.8}{NE}L\scalebox{0.8}{OOP}\xspace}
\newcommand{\FastJet}{F\scalebox{0.8}{AST}J\scalebox{0.8}{ET}\xspace}

\newcommand{\DIPHOX}{D\scalebox{0.8}{IPHOX}\xspace}
\newcommand{\MCFM}{M\scalebox{0.8}{CFM}\xspace}
\newcommand{\twogammaMC}{2\scalebox{0.8}{gammaMC}\xspace}
\newcommand{\VBFatNLO}{V\scalebox{0.8}{BF@NLO}\xspace}
\newcommand{\RESBOS}{R\scalebox{0.8}{ESBOS}\xspace}

\newcommand{\QCDpEW}{QCD+EW}
\newcommand{\QCDtEW}{QCD$\times$EW}
\newcommand{\deltaQCD}{\ensuremath{\delta_\text{QCD}}}
\newcommand{\deltaEW}{\ensuremath{\delta_\text{EW}}}
\newcommand{\muR}{\ensuremath{\mu_{\mathrm{R}}}}
\newcommand{\muF}{\ensuremath{\mu_{\mathrm{F}}}}

\newcommand{\comment}[1]{\textbf{#1}}

%%%%%%%%%%%%%%%%%%%%%%%%%%%%%%%%%%%%%%%%%%%%%%%%%%%%%%%%%%%%%%%%%%%%%%%%%%%%%%%
\usepackage{ulem}
\newcommand{\deleted}[1]{\sout{#1}}
% \newcommand{\deleted}[1]{}

\newcommand{\revised}[1]{\textcolor{blue}{#1}}
% \newcommand{\revised}[1]{#1}



%%%%%%%%%%%%%%%%%%%%%%%%%%%%%%%%%%%%%%%%%%%%%%%%%%%%%%%%%%%%%%%%%%%%%%%%%%%%%%%

%\begin{frontmatter}

\title{NLO QCD+EW corrections to diphoton production in association with a vector boson}

\author[a]{Nicolas Greiner,}
\author[a,b]{Marek Sch\"onherr}


\affiliation[a]{Physik Institut, Universit{\"a}t Z{\"u}rich, Winterthurerstr.190, 8057 Z\"urich, Switzerland}
\affiliation[b]{Theoretical Physics Department, CERN, 1211 Geneva 23, Switzerland}


\preprint{
  \small
  \begin{flushright}
    ZU--TH 31/17\\ CERN-TH-2017-226
  \end{flushright}
}

\abstract{
Processes with three external electroweak gauge boson allow for a measurement of triple and quartic gauge
couplings. They can be used to constrain anomalous gauge couplings, where new physics might predominantly
couple to electroweak gauge bosons. In this paper we chose a class of such processes where we consider two photons
and an additional vector boson in the final state. As additional vector boson we consider either a third photon or a
$W$ or $Z$ boson. For the latter two cases we assume a leptonic decay of the boson. We calculate the next-to-leading
order QCD and electroweak corrections to these processes with a particular emphasis on the until now unknown 
electroweak corrections. We find that the electroweak corrections to the total cross section are moderate in the range of a few 
per cent at most, but can reach several tens of per cent in regions of phase space  that are particularly interesting
 in the context of new physics searches. In addition we investigate the difference between additive and multiplicative scheme 
 when combining
QCD and electroweak corrections and we assess the importance of photon induced contributions to these processes. 
}
 

\keywords{EW corrections, Photon, NLO, Jets}

\hypersetup{
  pdftitle = {NLO QCD+EW corrections to diphoton production in association with a vector boson},
  pdfauthor = {Nicolas Greiner, Marek Schoenherr}
}


\begin{document}

\maketitle

%%%%%%%%%%%%%%%%%%%%%%%%%%%%%%%%%%%%%%%%%%%%%%%%%%%%%%%%%%%%%%%%%%%%%%%%%%%%%%%%
%% Introduction
%%%%%%%%%%%%%%%%%%%%%%%%%%%%%%%%%%%%%%%%%%%%%%%%%%%%%%%%%%%%%%%%%%%%%%%%%%%%%%%%
\section{Introduction}
\label{sec:intro}

The precise understanding of the electroweak symmetry breaking 
mechanism is an important cornerstone of the LHC physics 
programme. 
Different realisations of the electroweak sector or new physics 
coupling to electroweak gauge bosons will yield to deviations 
compared to the Standard Model prediction. 
New physics effects can conveniently be described in terms of 
an effective theory where new heavy degrees of freedom are 
integrated out and deviations from the Standard Model are 
parametrised by higher dimensional operators 
\cite{Weinberg:1978kz,Weinberg:1979pi}. 
Higher dimensional operators can lead to deviations in triple 
and quartic gauge couplings. Moreover, new vertices 
(e.g. $\gamma\gamma\gamma$, $Z\gamma\gamma$) that do not 
exist in the Standard Model can appear. 
The class of processes that involve a pair of photons in 
association with another vector boson allows to measure 
deviations in triple and quartic gauge couplings and is 
therefore a particularly interesting class of processes. 
Consequently both ATLAS and CMS have measured these types of 
processes and derived constraints on anomalous gauge couplings 
\cite{Aad:2016sau,Sirunyan:2017lvq,Aad:2015uqa,Aad:2015bua}. 

In this paper we calculate the next-to-leading order QCD and 
electroweak corrections to the processes $\gamma\gamma\gamma$, 
$\gamma\gamma e^{+} e^{-}$ and $\gamma\gamma e^{-} \bar{\nu_e}$. 
For the latter two, all possible off-shell contributions of 
intermediate vector bosons are taken into account. 
The QCD corrections to these processes have already appeared 
in the literature 
\cite{Bozzi:2011en,Campbell:2012ft,Bozzi:2011wwa}. 
For a more complete picture of the higher order effects we 
recompute them for a centre of mass energy of 
13\,TeV, and supplement them with the next-to-leading order 
electroweak corrections. 
Although the electroweak corrections are much smaller at the level 
of the total cross section compared to the NLO QCD corrections 
they are particularly important when deriving limits on 
anomalous couplings. 
The effects of higher dimensional operators increase in the 
high energy tails of differential distributions and lead to a 
change of the shapes. And it is in the same region where the 
Sudakov logarithms from the electroweak corrections will play 
an essential role as well.

The paper is organized as follows. 
In Section \ref{sec:setup} we describe the calculational setup 
that has been used to obtain our numerical result which we are 
going to discuss in Section \ref{sec:results} before we 
conclude in Section \ref{sec:conclusions}.
 


%%%%%%%%%%%%%%%%%%%%%%%%%%%%%%%%%%%%%%%%%%%%%%%%%%%%%%%%%%%%%%%%%%%%%%%%%%%%%%%%
%% Calculational Setup
%%%%%%%%%%%%%%%%%%%%%%%%%%%%%%%%%%%%%%%%%%%%%%%%%%%%%%%%%%%%%%%%%%%%%%%%%%%%%%%%
\section{Calculational setup}
\label{sec:setup}
The results presented in this paper have been obtained by combining the two tools
\GoSam~\cite{Cullen:2011ac,Cullen:2014yla} and \Sherpa~\cite{Gleisberg:2008ta}
which allows for a fully automated calculation of cross section and observables and next-to-leading order in QCD as well
as in the electroweak coupling.
\GoSam is a package which generates the code for the numerical evaluation of
the one loop scattering amplitudes starting from the Feynman diagrams,
generated with \QGraf~\cite{Nogueira:1991ex} and further processed with
\FORM~\cite{Vermaseren:2000nd,Kuipers:2012rf} and
\Spinney~\cite{Cullen:2010jv} to perform necessary algebraic
manipulations to obtain an optimized expression for the matrix elements.
For the integrand reduction of the diagrams we use the \Ninja
library~\cite{Peraro:2014cba}, an implementation of the technique of integrand
reduction via Laurent expansion~\cite{Mastrolia:2012bu,vanDeurzen:2013saa}.
Alternatively one can choose other reduction strategies such as OPP reduction
method~\cite{Ossola:2006us,Mastrolia:2008jb,Ossola:2008xq} which is
implemented in $d$ dimensions in \Samurai~\cite{Mastrolia:2010nb}, or methods based on
tensor integral reduction as implemented in
\GolemNF~\cite{Heinrich:2010ax,Binoth:2008uq,Cullen:2011kv,Guillet:2013msa}.
We have used \OneLoop~\cite{vanHameren:2010cp} to evaluate the scalar integrals.


We define our central scales through
\begin{equation}
  \label{eq:murfdef}
  \begin{split}
    \muR^0 = \muF^0 = \ldots
  \end{split}
\end{equation}
and subsequently vary them by the conventional factor 
of two to estimate higher order contributions.


\begin{figure}[t!]
  \begin{tabular}{ccccc}
    \includegraphics[width=0.288\textwidth]{diagrams/aaa_V_2} & &
    \includegraphics[width=0.288\textwidth]{diagrams/aaa_V_1} & &
    \includegraphics[width=0.288\textwidth]{diagrams/aaa_V_3} \\
  \end{tabular}
  \caption{
    Sample diagrams of electroweak virtual corrections to triple 
    photon production.
  }
\end{figure}

\begin{figure}[t!]
  \begin{tabular}{ccccc}
    \includegraphics[width=0.288\textwidth]{diagrams/aaW_V_2} & &
    \includegraphics[width=0.288\textwidth]{diagrams/aaW_V_1} & &
    \includegraphics[width=0.288\textwidth]{diagrams/aaW_V_3} \\
  \end{tabular}
  \caption{
    Sample diagrams of electroweak virtual corrections to diphoton 
    production in association with a lepton-neutrino pair.
  }
\end{figure}

\begin{figure}[t!]
  \begin{tabular}{ccccc}
    \includegraphics[width=0.288\textwidth]{diagrams/aaZ_V_2} & &
    \includegraphics[width=0.288\textwidth]{diagrams/aaZ_V_1} & &
    \includegraphics[width=0.288\textwidth]{diagrams/aaZ_V_3} \\
  \end{tabular}
  \caption{
    Sample diagrams of electroweak virtual corrections to diphoton 
    production in association with a lepton-pair.
  }
\end{figure}



%%%%%%%%%%%%%%%%%%%%%%%%%%%%%%%%%%%%%%%%%%%%%%%%%%%%%%%%%%%%%%%%%%%%%%%%%%%%%%%%
%% Results
%%%%%%%%%%%%%%%%%%%%%%%%%%%%%%%%%%%%%%%%%%%%%%%%%%%%%%%%%%%%%%%%%%%%%%%%%%%%%%%%
\section{Results}
\label{sec:results}

In this section we present numerical results for the NLO QCD and NLO EW 
corrections to all three production processes of a diphoton pair in 
association with a third vector boson, a third photon, a $W$ or a $Z$ 
boson, at the LHC at a centre-of-mass energy of 13\,TeV. 
In case of an accompanying $W$ or a $Z$ boson, we consider the full 
off-shell leptonic final state, i.e.\ lepton-neutrino or lepton-pair 
production.
All results are obtained in the Standard Model using the complex-mass 
scheme \cite{Denner:2005fg} with the following input parameters
\begin{center}
  \begin{tabular}{rclrcl}
    $\alphazero$ &\shortequal& $1/137.03599976$  \qquad &&& \\
    $G_\mu$ &\shortequal& $1.1663787\times 10^{-5}\; \text{GeV}^2$ &&& \\
    $m_W$ &\shortequal& $80.385\; \text{GeV}$       & $\Gamma_W$ &\shortequal& $2.085\; \text{GeV}$ \\
    $m_Z$ &\shortequal& $91.1876\; \text{GeV}$      & $\Gamma_Z$ &\shortequal& $2.4952\; \text{GeV}$ \\
    $m_h$ &\shortequal& $125.0\; \text{GeV}$        & $\Gamma_h$ &\shortequal& $0$\\
    $m_t$ &\shortequal& $173.2\; \text{GeV}$        & $\Gamma_t$ &\shortequal& $0$\;.
  \end{tabular}
\end{center}
Both the width of the top quark and the Higgs boson can safely be 
neglected as there are no diagrams containing either as $s$-channel 
propagators which can potentially go on-shell. 
All other lepton and parton masses and widths are set to zero, 
i.e.\ we are working in the five-flavour scheme.
We use the \textsc{CT14nlo} PDF set with $\alphas(m_Z)=0.118$. 
The use of a QCD-only PDF is justified by the fact that, 
at LO, the photon induced corrections are either non-existant 
(\aaa, \aaw) or negligible (\aaz).
This finding will be detailed in Section \ref{sec:results:aaz}.

We define our central scales through
\begin{equation}
  \label{eq:murfdef}
  \begin{split}
    \muR^0 = \muF^0 = \left\{
    \begin{array}{ll}
      m_{\aaa} \qquad & \text{in \aaa\ production} \\[2mm]
      \tfrac{1}{2}\,H_T' & \text{in \aaw\ and \aaz\ production.}
    \end{array}\right.
  \end{split}
\end{equation}
Therein, the modified scalar sum of transverse momenta is defined as 
\begin{equation}
  \label{ew:defHT}
  \begin{split}
    H_\mathrm{T}' = E_\mathrm{T}^V + \sum_{\gamma,q,g} p_{\mathrm{T},i}
  \end{split}
\end{equation}
with $\left.E_\mathrm{T}^W\right.^2=(p_\ell+p_\nu)^2$ and 
$\left.E_\mathrm{T}^Z\right.^2=(p_{\ell^+}+p_{\ell^-})^2$ 
in full analogy to the case of the case of vector boson production 
in association with jets \cite{}.
As the Born process in each case has no $\muR$ dependence, we 
do not expect the choice of scale to have a significant influence 
on the size of the relative QCD and EW corrections.
We calculate $\mathrm{B}(\muF)$, $\deltaQCD(\muR,\muF)$ and 
$\deltaEW(\muF^0)$ and vary the free scales $\muR$ and $\muF$ 
by the conventional factor of two around their central values 
$\muR^0$ and $\muF^0$, respectively.
We do not vary the factorisation scale for the determination 
of $\deltaEW$ as the inherent, albeit normally phenomenologically 
irrelevant, stabilisation of the $\muF$-dependence at NLO EW 
is not reflected in our chosen PDF.
Hence, our NLO EW result exhibits the exact same $\muF$-dependence 
as the LO result.


\begin{table}[t!]
  \centering
  \begin{tabular}{l||c|c|c}
      & $\;\;pp \to \gamma \gamma\gamma\;\;$
      & $\;\;pp \to \gamma \gamma e^-\bar\nu_e\;\;$ 
      & $\;\;pp \to \gamma \gamma e^+e^-\;\;$ \\
    \hline\hline
    $\sigma_\text{LO}\;\;[\text{pb}] $ &  &  & \\
    \hline
    $\delta_\text{QCD}\;\;[\text{pb}]\;\;\pTveto=\infty $ &  &  & \\
    \hline
    $\delta_\text{QCD}\;\;[\text{pb}]\;\;\pTveto=30\,\text{GeV} $ &  &  & \\
    \hline
    $\delta_\text{EW}\;\;[\text{pb}] $ &  &  & \\
  \end{tabular}
  \caption{
    Total cross sections at LO, NLO QCD and NLO EW for $\gamma\gamma\gamma$, 
    $\gamma\gamma e^-\bar\nu_e$ and $\gamma\gamma e^+e^-$
    production at 13\,TeV at the LHC.
    \label{tab:xsec}
  } 
\end{table}


\subsection[\texorpdfstring{$\gamma\gamma\gamma$}{aaa} production]
           {$\boldsymbol{\gamma\gamma\gamma}$ production}
\label{sec:results:aaa}

\begin{figure}[t!]
  \centering
  \includegraphics[width=0.32\textwidth]{figs_aaa/pT_y1}
  \includegraphics[width=0.32\textwidth]{figs_aaa/pT_y2}
  \includegraphics[width=0.32\textwidth]{figs_aaa/pT_y3}
  \caption{
    Transverse momentum of the leading (left), subleading (centre) 
    and third leading (right) photon 
    in triple photon production at the LHC at 13\,TeV. 
    The distributions are shown at LO (blue), NLO QCD (green), 
    NLO EW (orange), NLO \QCDpEW\ (red) and NLO \QCDtEW\ (black) 
    including scale uncertainties. The top ratio plot details 
    the relative corrections to the leading order cross section 
    without applying any jet veto, while the centre ratio plot 
    applies a jet veto of $p_\mathrm{T,jet}^\mathrm{veto}=\text{30\,GeV}$. 
    The lower ratio highlights the size of the electroweak corrections.
    \label{fig:aaa:pt}
  }
\end{figure}

\begin{figure}[t!]
  \centering
  \includegraphics[width=0.32\textwidth]{figs_aaa/m_y1y2_comb_log}
  \includegraphics[width=0.32\textwidth]{figs_aaa/m_y1y3_comb_log}
  \includegraphics[width=0.32\textwidth]{figs_aaa/m_y2y3_comb_log}
  \caption{
    Pairwise invariant mass of the leading and subleading photon (left),
    leading and third leading photon (centre), subleading and third leading 
    photon (right) 
    in triple photon production at the LHC at 13\,TeV. 
    Details as in Fig.\ \ref{fig:aaa:pt}.\\
    \comment{MS: Spike in $m_{\gamma_1\gamma_2}$ is in NLO \QCDtEW\ 
             and comes from $\deltaEW\approx 10$ in this bin at the 
             edge of the LO phase space.}
    \label{fig:aaa:myy}
  }
\end{figure}

\begin{figure}[t!]
  \centering
  \includegraphics[width=0.32\textwidth]{figs_aaa/dphi_y1y2}
  \includegraphics[width=0.32\textwidth]{figs_aaa/dphi_y1y3}
  \includegraphics[width=0.32\textwidth]{figs_aaa/dphi_y2y3}
  \caption{
    Azimuthal separation of the leading and subleading photon (left),
    leading and third leading photon (centre), subleading and third leading 
    photon (right) 
    in triple photon production at the LHC at 13\,TeV. 
    Details as in Fig.\ \ref{fig:aaa:pt}.\\
    \comment{MS: Spike in $\Delta\phi_{\gamma_2\gamma_3}$ is in NLO \QCDtEW\ 
             and comes from $\deltaEW\approx 2$ in this bin at the 
             edge of the LO phase space.}
    \label{fig:aaa:dphi}
  }
\end{figure}

\begin{itemize}
  \item \pT of 3rd photon much softer than 1st and 2nd
  \item EW corrections to \pT of 1st and 2nd photon nearly identical and 
        $\approx -10\%$ at 500\,GeV, 3rd photon almost twice that.
  \item in all cases NLO EW outside LO uncertainty band
  \item NLO QCD dominated by real corrections, i.e.\ additional 
        jet emissions
  \item intensified through opening of new $qg$ channels and large 
        gluon luminosity
  \item can be controlled in fixed-order calculations through jet veto, 
        needs proper multijet merging
  \item because at LO the leading jet needs to be in the opposite 
        hemisphere as the subleading and third leading jet, 
        $m_{\gamma_1\gamma_2}$ and $m_{\gamma_1\gamma_3}$ 
        exhibit kinematic edges at LO
  \item kinematic conditions relaxed at NLO (both QCD and EW, 
        but QCD dominating), large corrections around those edges, 
        leads to artifacts in multiplicative combination
  \item NLO QCD very similar for $m_{\gamma_1\gamma_2}$ and 
	$m_{\gamma_1\gamma_3}$, $m_{\gamma_2\gamma_3}$ with interesting 
	structure around 80-90\,GeV
  \item $m_{\gamma_1\gamma_2}$ again exhibits small excess at $2m_W$
  \item small positive EW corrections at small $m_{\gamma\gamma}$ 
  \item $\approx -10\%$ at 1\,TeV for all $m_{\gamma\gamma}$ 
  \item NLO EW very similar for all $m_{\gamma\gamma}$ 
\end{itemize}



\subsection[\texorpdfstring{$\gamma\gamma\ell\nu$}{aalnu} production]
           {$\boldsymbol{\gamma\gamma\ell\nu}$ production}
\label{sec:results:aaw}

\comment{MS: is that $\ell^+$ or $\ell^-$?}

\begin{figure}[t!]
  \centering
  \includegraphics[width=0.32\textwidth]{figs_aaw/pT_y1}
  \includegraphics[width=0.32\textwidth]{figs_aaw/pT_y2}
  \includegraphics[width=0.32\textwidth]{figs_aaw/pT_l1}
  \caption{
    Transverse momentum of the leading (left), subleading (centre) 
    and third leading (right) photon at the LHC at 13\,TeV.
    \label{fig:aaw:pt}
  }
\end{figure}

\begin{figure}[t!]
  \centering
  \includegraphics[width=0.32\textwidth]{figs_aaw/m_y1y2_comb_log}
  \includegraphics[width=0.32\textwidth]{figs_aaw/m_y1l1_comb_log}
  \includegraphics[width=0.32\textwidth]{figs_aaw/m_t_comb_log}
  \caption{
    Pairwise invariant mass of the leading and subleading photon (left),
    leading and third leading photon (centre), subleading and third leading 
    photon (right) at the LHC at 13\,TeV.\\
    \comment{MS: Spike in $m_{\gamma_1\gamma_2}$ is in NLO \QCDtEW 
             and comes from $\deltaEW\approx 10$ in this bin at the 
             edge of the LO phase space.}
    \label{fig:aaw:myy}
  }
\end{figure}

\begin{figure}[t!]
  \centering
  \includegraphics[width=0.32\textwidth]{figs_aaw/dphi_y1_y2}
  \includegraphics[width=0.32\textwidth]{figs_aaw/dphi_y1_l1}
  \includegraphics[width=0.32\textwidth]{figs_aaw/dphi_y2_l1}
  \caption{
    Azimuthal separation of the leading and subleading photon (left),
    leading and third leading photon (centre), subleading and third leading 
    photon (right) at the LHC at 13\,TeV.\\
    \comment{MS: Spike in $m_{\gamma_1\gamma_2}$ is in NLO \QCDtEW 
             and comes from $\deltaEW\approx 10$ in this bin at the 
             edge of the LO phase space.}
    \label{fig:aaw:dphi}
  }
\end{figure}


\subsection[\texorpdfstring{$\gamma\gamma\ell^+\ell^-$}{aall} production]
           {$\boldsymbol{\gamma\gamma\ell^+\ell^-}$ production}
\label{sec:results:aaz}

\begin{figure}[t!]
  \centering
  \includegraphics[width=0.32\textwidth]{figs_aaz/pT_y1}
  \includegraphics[width=0.32\textwidth]{figs_aaz/pT_y2}
  \includegraphics[width=0.32\textwidth]{figs_aaz/pT_l1l2_comb_log}
  \caption{
    Transverse momentum of the leading (left) and subleading (centre) 
    photon as well as the dressed lepton pair (right) 
    in diple photon production in association with a lepton pair 
    at the LHC at 13\,TeV. 
    Details as in Fig.\ \ref{fig:aaa:pt}.
    \label{fig:aaz:pt}
  }
\end{figure}

\begin{figure}[t!]
  \centering
  \includegraphics[width=0.32\textwidth]{figs_aaz/m_y1y2_comb_log}
  \includegraphics[width=0.32\textwidth]{figs_aaz/m_y1l1l2_comb_log}
  \includegraphics[width=0.32\textwidth]{figs_aaz/m_y2l1l2_comb_log}
  \caption{
    Pairwise invariant mass of the leading and subleading photon (left),
    the leading photon and the lepton pair (centre), and the subleading 
    photon and the lepton pair (right)
    in diple photon production in association with a lepton pair 
    at the LHC at 13\,TeV. 
    Details as in Fig.\ \ref{fig:aaa:pt}.\\
    \comment{MS: Spike in $m_{\gamma_1\ell_1\ell_2}$ is in NLO \QCDtEW\ 
             and comes from $\deltaEW\approx 10$ in this bin at the 
             edge of the LO phase space.}
    \label{fig:aaz:myy}
  }
\end{figure}

\begin{figure}[t!]
  \centering
  \includegraphics[width=0.32\textwidth]{figs_aaz/dphi_y1_y2}
  \includegraphics[width=0.32\textwidth]{figs_aaz/dphi_y1_l1l2}
  \includegraphics[width=0.32\textwidth]{figs_aaz/dphi_y2_l1l2}
  \caption{
    Azimuthal separation of the leading and subleading photon (left),
    the leading photon and the lepton pair (centre), and the subleading 
    photon and the lepton pair (right)
    in diple photon production in association with a lepton pair 
    at the LHC at 13\,TeV. 
    Details as in Fig.\ \ref{fig:aaa:pt}.\\
    \comment{MS: Spike in $\Delta\phi_{\gamma_1,\ell_1\ell_2}$ is in NLO \QCDtEW\ 
             and comes from $\deltaEW\approx 10$ in this bin at the 
             edge of the LO phase space.}
    \label{fig:aaz:dphi}
  }
\end{figure}



\begin{figure}[t!]
  \setlength{\unitlength}{\textwidth}
  \begin{picture}(0,0.37)
    \put(0,0.24){\includegraphics[width=0.32\textwidth]{figs_aaz_aa-ind/pT_y1}}
    \put(0,0.12){\includegraphics[width=0.32\textwidth]{figs_aaz_aa-ind/pT_y2}}
    \put(0,0){\includegraphics[width=0.32\textwidth]{figs_aaz_aa-ind/pT_l1l2_comb_log}}
    \put(0.33,0.24){\includegraphics[width=0.32\textwidth]{figs_aaz_aa-ind/m_y1y2_comb_log}}
    \put(0.33,0.12){\includegraphics[width=0.32\textwidth]{figs_aaz_aa-ind/m_y1l1l2_comb_log}}
    \put(0.33,0){\includegraphics[width=0.32\textwidth]{figs_aaz_aa-ind/m_y2l1l2_comb_log}}
    \put(0.66,0.24){\includegraphics[width=0.32\textwidth]{figs_aaz_aa-ind/dphi_y1_y2}}
    \put(0.66,0.12){\includegraphics[width=0.32\textwidth]{figs_aaz_aa-ind/dphi_y1_l1l2}}
    \put(0.66,0){\includegraphics[width=0.32\textwidth]{figs_aaz_aa-ind/dphi_y2_l1l2}}
  \end{picture}
  \caption{
    Contribution of $\gamma\gamma$-induced production channels at LO.
    \label{fig:aaz:aa-ind}
  }
\end{figure}





%%%%%%%%%%%%%%%%%%%%%%%%%%%%%%%%%%%%%%%%%%%%%%%%%%%%%%%%%%%%%%%%%%%%%%%%%%%%%%%%
%% Conclusions
%%%%%%%%%%%%%%%%%%%%%%%%%%%%%%%%%%%%%%%%%%%%%%%%%%%%%%%%%%%%%%%%%%%%%%%%%%%%%%%%
\section{Conclusions}
\label{sec:conclusions}

We did good.




 

\section*{Acknowledgements}
N.G.\ was supported by the Swiss National Science Foundation under contract
PZ00P2\_154829. M.S.\ was supported by PITN--GA--2012--315877 ({\it MCnet}) 
and the ERC Advanced Grant MC@NNLO (340983).

%%%%%%%%%%%%%%%%%%%%%%%%%%%%%%%%%%%%%%%%%%%%%%%%%%%%%%%%%%%%%%%%%%%%%%%%%%%%%%%%


\bibliographystyle{amsunsrt_mod}
\bibliography{journal}

\end{document}

%%%%%%%%%%%%%%%%%%%%%%%%%

